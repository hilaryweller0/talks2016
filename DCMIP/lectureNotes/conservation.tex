
\chapter{Conservation}
\begin{itemize}
\item What should be conserved? \citep[Material from ][]{Thu08a,RTSK10}
\item How can conservation be achieved?
\end{itemize}

\section{Should Weather and Climate Models Conserve Mass?}

\cleardoublepage{}

\liststepwise{
\begin{itemize}
\item \step{ECMWF and the Met Office are the most skillful weather forecasting
models and they do \textit{not} conserve mass. }\step{Why not?

\begin{itemize}
\item \step{They rely on semi-Lagrangian advection to achieve reasonable
time-steps on lat-lon grids}
\end{itemize}

}\step{However we are moving away from lat-lon grids because they
do not work well on massively parallel computers

\item }\step{Mass is conserved in the real atmosphere over all time-scales,
regardless of adiabatic processes and friction (neglecting relativistic
effects)
\item }\step{Mass errors $\rightarrow$ pressure errors $\rightarrow$
spurious winds
\item }\step{Failure to conserve mass means that nothing else can be conserved.
\item }\step{You do not want to lose all the mass in the atmosphere over
a long climate simulation}
\end{itemize}
\noindent \begin{center}
\step{\textbf{So }\textbf{\textit{YES}}\textbf{ weather and climate
models should conserve mass}}
\par\end{center}

}


\section{Should Weather and Climate Models Conserve Energy?}

\cleardoublepage{}

\liststepwise{
\begin{itemize}
\item \step{Energy is difficult to conserve because it consists of many
different types of energy which are calculated separately and transfers
between them may not be conservative. It consists of:}

\begin{itemize}
\item \step{Available and un-available potential energy (the unavailable
potential energy is the potential energy that the atmosphere would
have if it were in stationary hydrostatic balance)
\item }\step{kinetic energy 
\item }\step{internal energy (calculated from temperature)
\item }\step{chemical energy}
\end{itemize}
\item \step{Unavailable PE is much larger than all the others and does
not undergo a cascade to smaller scales. $\therefore$ should be conserved
exactly.
\item }\step{Available PE has a time-scale of $\approx$20 days in the
atmosphere so spurious sources should lead to slower changes than
this.
\item }\step{Kinetic energy cascades to small scales and there will always
be unresolved kinetic energy. $\therefore$ models should dissipate
rather than conserve KE (which should lead to a rise in temperature).
\item }\step{Formal energy conservation helps with model stability.}
\end{itemize}
\noindent \begin{center}
\step{\textbf{Avoid large spurious sinks and especially spurious
}\textbf{\textit{sources}}\textbf{ of energy}}
\par\end{center}

}


\section{Should Weather and Climate Models Conserve Momentum?}

\cleardoublepage{}

\liststepwise{
\begin{itemize}
\item \step{There is a transfer of momentum with the Earth's surface so
momentum is not conserved in the atmosphere.
\item }\step{Momentum is not dissipated -- there is no cascade to small
scales.
\item }\step{Momentum is conserved over very long time-scales in the stratosphere.
\item }\step{Most models do not conserve momentum
\item }\step{The stratospheric quasi-biennial oscillation is very difficult
to simulate. Related?}
\end{itemize}
\noindent \begin{center}
\step{\textbf{Don't know. Usually more focus on energy, enstrophy
and vorticity in atmospheric modelling.}}
\par\end{center}

}


\section{Should Weather and Climate Models Conserve Tracer Variance and Potential
Enstrophy?}

\liststepwise{

What is potential enstrophy? ... \opttext{Variance of the potential
vorticity:
\[
\text{potential enstrophy}=\frac{1}{2}hq^{2}
\]
where
\[
\text{potential vorticity, }q=(\boldsymbol{\eta}+2\boldsymbol{\Omega})\cdot\mathbf{k}/h\ \text{, vorticity, }\boldsymbol{\eta}=\nabla\times\mathbf{u}
\]
}}

\cleardoublepage{}

\liststepwise{
\begin{itemize}
\item \step{Cascade to small scales then dissipation in about 10 days
\item }\step{Lots of un-resolved tracer variance and enstrophy
\item }\step{Conservation $\rightarrow$ spectral blocking at the grid-scale 
\item }\step{$\therefore$ need to make sure that tracer variance and enstrophy
are destroyed at the grid scale. How ... \step{

\begin{itemize}
\item Scale selective dissipation or
\item Dissipative advection scheme, eg

\begin{itemize}
\item Odd order finite volume with a limiter
\item Even order semi-Lagrangian
\end{itemize}

So that leading order error is dissipative rather than dispersive

\end{itemize}

}}

\end{itemize}
}


\section{Should Weather and Climate Models Conserve Potential Vorticity?}

\cleardoublepage{}

\liststepwise{
\begin{itemize}
\item \step{PV conservation $\rightarrow$ correct strength of weather
systems
\item }\step{PV is derived from velocity so conservation requires careful
numerics}
\end{itemize}
}

\clearpage{}


\section{How can Mass be Conserved?}

\liststepwise{

Air density, $\rho$, satisfies the continuity equation:
\begin{equation}
\frac{\partial\rho}{\partial t}+\nabla\cdot(\mathbf{u}\rho)=0\label{eq:cont}
\end{equation}
where $\mathbf{u}$ is the wind. This is a conservation equation for
$\rho$. \step{Let us assume that $\rho$ and $\mathbf{u}$ vary
only in the $x$ direction and that our periodic domain goes from
$x=0$ to $x=2\pi$ (for example around the equator). Then the total
mass of air is:}\step{
\begin{equation}
M=\int_{0}^{2\pi}\rho\ dx\label{eq:mass}
\end{equation}
}\step{If we solve eqn (\ref{eq:cont}) using a finite volume scheme
we get:
\begin{equation}
\rho_{j}^{(n+1)}=\rho_{j}^{(n)}-\opttext{\Delta t\frac{u_{j+1/2}\rho_{j+1/2}-u_{j-1/2}\rho_{j-1/2}}{\Delta x}}\label{eq:oneDFV}
\end{equation}
where $u_{j\pm1/2}$ and $\rho_{j\pm1/2}$ can be defined in any way
from the values of $u_{j}$ and $\rho_{j}$ at the grid points (eg
interpolation). }\step{The total mass in the computational domain
at time $n$ is:
\begin{equation}
M^{(n)}=\opttext{\sum_{j=1}^{j=N}\Delta x\rho_{j}^{(n)}.}\label{eq:compMass}
\end{equation}
}\step{To find $M^{(n+1)}$ as predicted from the finite volume scheme,
we can substitute eqn (\ref{eq:oneDFV}) into (\ref{eq:compMass}):}

}

\clearpage{}\liststepwise{
\begin{eqnarray}
M^{(n+1)} & = & \sum_{j=1}^{j=N}\Delta x\rho_{j}^{(n+1)}=\step{\sum_{j=1}^{j=N}\left\{ \Delta x\rho_{j}^{(n)}-\Delta t\left(u_{j+1/2}\rho_{j+1/2}-u_{j-1/2}\rho_{j-1/2}\right)\right\} }\\
 & = & \opttext{M^{(n)}-\Delta t\left(\sum_{j=1}^{j=N}u_{j+1/2}\rho_{j+1/2}-\sum_{j=1}^{j=N}u_{j-1/2}\rho_{j-1/2}\right)}\\
 & = & \step{M^{(n)}-\Delta t\left(\sum_{j=1}^{j=N}u_{j+1/2}\rho_{j+1/2}-\sum_{j=0}^{j=N-1}u_{j+1/2}\rho_{j+1/2}\right)}\\
 & = & \opttext{M^{(n)}-\Delta t\left(u_{N+1/2}\rho_{N+1/2}-u_{1/2}\rho_{1/2}\right)}
\end{eqnarray}


\step{and since we have periodic boundary conditions, $u_{N+1/2}\rho_{N+1/2}=u_{1/2}\rho_{1/2}$
proving that $M^{(n+1)}=M^{(n)}$ and hence mass is conserved.

}\step{If instead we discretised the advective form of the continuity
equation:
\begin{equation}
\frac{\partial\rho}{\partial t}+\mathbf{u}\cdot\nabla\rho+\rho\nabla\cdot\mathbf{u}=0\label{eq:contAdv}
\end{equation}
using finite differences or semi-Lagrangian, conservation would not
be so easy to achieve.}

}

\clearpage{}


\section{How can Energy be Conserved?}

\liststepwise{

One possible technique would be to solve a conservation equation for
total energy and then calculate temperature from the energy. However
this is not done in atmospheric models. Why not? ... \step{Instead
we can consider how energy can be conserved in the vector invariant
form of the linearised shallow-water equations, solving for $\mathbf{u}$
and $h$:}\step{
\begin{align}
\frac{\partial\mathbf{u}}{\partial t} & +\boldsymbol{\eta}\times\mathbf{u}=-g\nabla h-\nabla K\label{eqn:linSWEuv}\\
\frac{\partial h}{\partial t} & +\nabla\cdot(h\mathbf{u})=0\label{eqn:linSWEh}
\end{align}
where $\boldsymbol{\eta}=\nabla\times\mathbf{u}+2\boldsymbol{\Omega}$
is the total vorticity and $K={\textstyle \frac{1}{2}|\mathbf{u}|^{2}}$
is the kinetic energy. }\step{The energy is defined as:
\begin{equation}
E=\frac{1}{2}gh^{2}+hK.
\end{equation}
}\step{If we calculate $h\mathbf{u}\cdot$(\ref{eqn:linSWEuv})$+(gh+K)$
(\ref{eqn:linSWEh}) we can derive the conservation equation for $E$:
\begin{equation}
\opttext{h\mathbf{u}\cdot\frac{\partial\mathbf{u}}{\partial t}+h\mathbf{u}\cdot(\boldsymbol{\eta}\times\mathbf{u})+(gh+K)\frac{\partial h}{\partial t}+(gh+K)\nabla\cdot(h\mathbf{u})=-gh\mathbf{u}\cdot\nabla h-h\mathbf{u}\cdot\nabla K}\label{eq:linSWenergy}
\end{equation}


}}

\clearpage{}

\liststepwise{

Using the vector calculus identities:
\begin{eqnarray*}
\mathbf{u}\cdot(\boldsymbol{\eta}\times\mathbf{u}) & = & 0\\
K\nabla\cdot(h\mathbf{u})+h\mathbf{u}\cdot\nabla K & = & \nabla\cdot(Kh\mathbf{u})
\end{eqnarray*}
eqn (\ref{eq:linSWenergy}) can be rearranged to give:
\begin{equation}
\frac{\partial E}{\partial t}+\opttext{\nabla\cdot(h^{2}\mathbf{u}+hK\mathbf{u})=0.}
\end{equation}
\step{The term $\nabla\cdot(h^{2}\mathbf{u}+hK\mathbf{u})$ is the
divergence of a flux so it cannot create or destroy energy, just move
energy around. Therefore, for a SWE model to conserve energy, it must
have discrete versions of the above vector calculus identities. These
are called \textit{mimetic }properties. The mimic properties of the
real system.}

}\clearpage{}


\section{How can Potential Vorticity be Conserved?}

\liststepwise{

The PV equation can be derived by taking the curl of the momentum
equation (in vector invariant form):
\begin{equation}
\nabla\times\frac{\partial\mathbf{u}}{\partial t}+\nabla\times(\boldsymbol{\eta}\times\mathbf{u})=\opttext{-\cancelto{0}{\nabla\times(g\nabla h+\nabla K)}}
\end{equation}
\step{Considering the 2D flow of the SWE and uing the vector calculus
identities:
\begin{eqnarray*}
\nabla\times\nabla h & = & 0
\end{eqnarray*}
this gives:
\[
\frac{\partial hq}{\partial t}+\nabla\cdot(qh\mathbf{u})=0
\]
which is a conservation equation for the PV, $q=(\boldsymbol{\eta}+2\boldsymbol{\Omega})\cdot\mathbf{k}/h$.}\step{
So for a model to conserve PV, it must have a discrete equivalent
of curl free gradients. This is another \textit{mimetic }property.}

}
