\cleardoublepage{}


\chapter{Semi-Implicit Time-Stepping}


\subsection*{Recap Some Advantages \textmd{(and implied disadvantages) of aspects
of time-stepping}}

\liststepwise{

\begin{tabular}{|>{\centering}p{0.48\textwidth}|>{\centering}p{0.48\textwidth}|}
\hline 
\textbf{Implicit} & \textbf{Explicit}\tabularnewline
\hline 
\step{Stable for Courant number $>>1$}\vspace{10pt}
 & \step{Cheap}\vspace{10pt}
\tabularnewline
\hline 
\hline 
\textbf{Short time-steps} & \textbf{Long time-steps}\tabularnewline
\hline 
\step{Accurate}\vspace{10pt}
 & \step{Cheap}\vspace{10pt}
\tabularnewline
\hline 
\end{tabular}


\subsection*{Modelling Considerations}

\begin{minipage}[t]{0.48\columnwidth}%
We can afford this setup for our model:

\begin{tabular}{|c|c|}
\hline 
$\Delta x$, $\Delta y$ & 10km\tabularnewline
\hline 
$\Delta z$ & 200m\tabularnewline
\hline 
$\Delta t$ & 1 minute\tabularnewline
\hline 
\end{tabular}%
\end{minipage} %
\begin{minipage}[t]{0.48\columnwidth}%
The atmosphere has the following speeds:

\begin{tabular}{|l|>{\centering}m{0.3\textwidth}|}
\hline 
Acoustic wave speed & \opttext{340m/s}\tabularnewline
\hline 
Gravity wave speed & \opttext{$\le$300m/s}\tabularnewline
\hline 
Horizontal wind & \opttext{$\le$80m/s}\tabularnewline
\hline 
Vertical wind & \opttext{$\le$1m/s}\tabularnewline
\hline 
\end{tabular}%
\end{minipage}

What Courant numbers will these speeds and resolutions lead to? What
are the implications for choices of explicit and implicit time-stepping
(assuming that explicit schemes are typically stable for Courant numbers
less than one)?}\clearpage{}

\begin{tabular}{|>{\raggedright}p{0.15\textwidth}|>{\centering}p{0.2\textwidth}|>{\raggedright}p{0.55\textwidth}|}
\hline 
 & \textbf{Maximum Courant number} & \textbf{Implication}\tabularnewline
\hline 
\hline 
\textbf{Horizontal acoustic } & \lecText{$340\times60/10^{3}\approx2$} & \lecText{Treat horizontal acoustic waves implicitly, use a shorter
time-step or use sub-stepping (split-explicit)}\tabularnewline
\hline 
\textbf{Vertical acoustic} & \lecText{$340\times60/200\approx100$} & \lecText{Vertically propaganing acoustic waves must be treated implicitly}\tabularnewline
\hline 
\textbf{Gravity wave} & \lecText{$300\times60/10^{3}\approx2$} & \lecText{Treat gravity waves implicitly, use a shorter time-step
or use sub-stepping (split-explicit)}\tabularnewline
\hline 
\textbf{Horizontal wind} & \lecText{$80\times60/10^{3}\approx0.5$} & \lecText{Can be treated explicitly (unless you are on a lat-lon grid)}\tabularnewline
\hline 
\textbf{Vertical Wind} & \lecText{$1\times60/200\approx0.3$} & \lecText{Can be treated explicitly as long as you do not try to resolve
convection or other processes which will increase the maximum vertical
wind.}\tabularnewline
\hline 
\end{tabular}

\clearpage{}

Treating gravity and acoustic wave implicitly and advection (and possibly
Coriolis) explicitly is called \textbf{semi-implicit}. For the shallow-water
equations, if we treat gravity waves using backward Euler and advection
and Coriolis with forward Euler, mark which terms are evaluated at
time-level $n$ and which at time-level $n+1$.\liststepwise{
\begin{align}
\frac{\mathbf{u}^{(n+1)}-\mathbf{u}^{(n)}}{\Delta t}+\mathbf{u}^{(\opttext{n\ \ })}\cdot\nabla\mathbf{u}^{(\opttext{n\ \ })} & =-2\Omega\times\mathbf{u}^{(\opttext{n\ \ })}-g\nabla h^{(\opttext{n+1})}\label{eqn:SWEuvtime}\\
\frac{h^{(n+1)}-h^{(n)}}{\Delta t}+\mathbf{u}^{(\opttext{n\ \ })}\cdot\nabla h^{(\opttext{n\ \ })} & =-h^{(\opttext{n\ \ })}\nabla\cdot\mathbf{u}^{(\opttext{n+1})}\label{eqn:SWEhtime}
\end{align}
This is solved by rearranging equation \ref{eqn:SWEuvtime} for $u^{(n+1)}$
and substituting these into equation \ref{eqn:SWEhtime}. This leads
to a semi-discretised version of the wave equation (\ref{eq:linSWEh})
or Helmholtz equation which can be solved for $h^{(n+1)}$. 
\begin{align*}
\mathbf{u}^{(n+1)} & =\opttext{\mathbf{u}^{(n)}-\Delta t\bigl(\mathbf{u}^{(n)}\cdot\nabla\mathbf{u}^{(n)}+2\Omega\times\mathbf{u}^{(n)}+g\nabla h^{(n+1)}\bigr)}
\end{align*}
\begin{align*}
\implies & \frac{h^{(n+1)}-h^{(n)}}{\Delta t}+\mathbf{u}^{(n)}\cdot\nabla h^{(n)}\\
 & =-h^{(n)}\nabla\cdot\biggr(\opttext{\mathbf{u}^{(n)}-\Delta t\bigl(\mathbf{u}^{(n)}\cdot\nabla\mathbf{u}^{(n)}+2\Omega\times\mathbf{u}^{(n)}+g\nabla h^{(n+1)}\bigr)}\biggr)
\end{align*}
The terms that make this into a Helmholtz equation are: 
\[
\frac{h^{(n+1)}-h^{(n)}}{\Delta t}=h^{(n)}\nabla\cdot\biggr(\Delta tg\nabla h^{(n+1)}\biggr)+\ldots=\Delta tgh^{(n)}\nabla^{2}h^{(n+1)}+\ldots
\]
which must be solved implicitly since $h^{(n+1)}$ appears on the
RHS and LHS.

A similar procedure is used to solve the Navier-Stokes equations in
semi-implicit models of the global atmosphere.}
